\begin{center}
  \section*{IT Automation with OpenStack and Ansible/AWX}
\end{center}


\begin{flushright}
  \begin{itemize}
  \item   登壇者: レッドハット株式会社 斉藤 秀喜
  \item 資料: 
  \end{itemize}
\end{flushright}


\section*{概要}

Ansible/AWXは構成管理ツールAnsibleに不足している機能を提供するものである。
提供機能は下記の通りである。

\begin{itemize}
\item Webフロント
\item ホストの認証管理
\item 実行ユーザーの管理
\item ログの一元管理
\item ワークフローの定義
\item 外部機能との連携API
\end{itemize}

AWXはコンテナで構成されKubernetesで構築されており、
スケールアウトを簡便に実現することができる。

\section*{詳細}

構成管理ツールAnsibleはホストへ定義したタスクを実行するツールである。
もともとタスクを実行することにのみ特化したツールであるため、
Ansible単体ではIaCのCI/CDを実現する統合ITオートメーションを構築できない。
そこで、Ansible AWXの開発がすすめられている。

Ansible AWXはAnsible単体で実現できなかった次のことを実現する。

\begin{itemize}
\item 実行ユーザーの管理
\item 実行ログの収集
\item ワークフローの定義
\item キャパシティー管理
\item 外部システムとの連携用API
\item IaaS連携
\item リビジョン管理
\item Web UI
\end{itemize}

