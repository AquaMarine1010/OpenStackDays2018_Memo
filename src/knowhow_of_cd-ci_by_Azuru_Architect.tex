\begin{center}
  \section*{Azureのアーキテクトが語るインフラCI/CDの勘所}
\end{center}


\begin{flushright}
  \begin{itemize}
  \item   登壇者: 日本マイクロソフト株式会社 真壁 徹
  \item 資料: 
  \end{itemize}
\end{flushright}

\section*{概要}

Infrastracture as Code(IaC)によるシステムの運用は、システムをコードに置き換えて設定値を管理するのみではうまくいかない。
システムをIaCに即するように設計しなおし構築する必要がある。
また、IaCを導入しCI/CDを実現するためには、Gitやその他コードを書くスキルが要求される。
チームメンバーには技術力が必要とされ、PMはスキルを満す要員確保能力を必要とされる。
それらを満すことができなければ、IaCをおこないCI/CDによるインフラ管理は無意味となる。

\section*{詳細}

IaCは構成管理ツールでインフラをコード化し、バージョン管理ツールでコードを管理し、
コードを継続的インテグレーションツールで構成をデプロイすることでCI/CDを実現する。
IaCを実現すると、インフラ管理が人手でによる管理から開放される。
クラウドで事前にコードをテストするようにCI/CDのパイプラインを設計すると、
コードの実行は必ず成功し常に意図した状態にインフラを保つことができる。

IaCはインフラ管理の万能薬として語られているが、適用した現場では様々な問題が発生している。\\

\begin{itemize}
\item 人員の確保
\item 羃等性の担保
\item 変更の困難さ
\item 困難なコードのテスト
\end{itemize}

IaCによるインフラの管理は、インフラをコードで制御するためコード開発の知識が必要になる。
また、インフラ管理者はコードのコード規約、コミット方法、リポジトリの戦略、ブランチルール、テスト環境など
開発全般の知識を考慮に入れCI/CDのパイプラインを設計し運用していかなければならない。\\

さらに、パイプラインは運用だけでなく開発の領域にも含まれる。
例えば、アプリケーションをコンテナに内包しk8sでデプロイする場合、
コンテナの定義ファイルの作成は開発者の責任範囲となる。
そうしたとき開発者はパイプラインに沿った開発を必要とする。
つまり、IaCを適用するしCI/CDでインフラ管理する組織は、開発・運用が相互の知識を必要としパイプラインを改善しなければならない。
組織のプロジェクトマネージャは適合する人員を定期的に確保する能力が問われる。
人員確保が困難な場合はIaCによるインフラ管理を諦めざるをえない。\\

IaCの羃等性担保の困難さについてである。
構成管理ツールがIaCの羃等性を担保している。
しかし、構成管理ツールのAnsible,Chel,Puppet,SaltはOSSであり、
すべてのコードを理解することは困難である。
そのため、運用者が羃等性を担保するのは困難となる。\\


システムの変更は、構成管理ツールの設定値を更新することで実現する。
しかし、往々にして設定値の変更による更新は失敗する。
モジュールが羃等性に対応していなかったり、
ログなどの管理外のデータなどによる依存による失敗である。
対策としては、システムを常に新規構築するように設計する。\\
例えば、ステートレスとステートフルの状態に分け、ステートレスな部分は毎回破棄し作成するような運用する。
ステートフルな箇所は手作業にして対策する。\\

コードのテストは、ルーターなどテスト不可能な箇所が出てくる。
IaCはテスト前提なため、テストできない箇所に関しては手動でおこなう必要がある。\\

IaCにより人手によるインフラ運用から開放されることは今のところないが、
工夫次第により工数は削減可能である。
しかし、実現には組織全体での取り組み、プロジェクトマネージャの人員確保のスキルを必要とする。
